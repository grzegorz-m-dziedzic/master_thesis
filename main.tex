%\documentclass[twoside]{pwrthesis}
\documentclass[twoside]{iisthesis}
% ---
\usepackage{polski}
\usepackage[utf8]{inputenc}
\usepackage{amsmath}
\usepackage{tocloft}
\usepackage{listings}
\usepackage{algorithm}
\usepackage{algorithmic}
\usepackage{subcaption}
\usepackage{mathtools}
\usepackage{graphicx}
\usepackage[colorinlistoftodos]{todonotes}
\usepackage{url}
\usepackage{pgfplots, pgfplotstable}
\selectlanguage{polish}
% Dodane przeze mnie d
\usepackage{fancyvrb} % dla srodowiska Verbatim
\usepackage{color}
\usepackage{lscape}

\hypersetup{
    colorlinks,
    linkcolor={black!50!black},
    citecolor={black!50!black},
    urlcolor={black!80!black}
}

\definecolor{gray}{rgb}{0.4,0.4,0.4}
\definecolor{darkblue}{rgb}{0.0,0.0,0.6}
\definecolor{cyan}{rgb}{0.0,0.6,0.6}

\lstset{
  basicstyle=\ttfamily,
  columns=fullflexible,
  showstringspaces=false,
  commentstyle=\color{gray}\upshape
}

\lstdefinelanguage{XML}
{
  morestring=[b]",
  morestring=[s]{>}{<},
  morecomment=[s]{<?}{?>},
  stringstyle=\color{black},
  identifierstyle=\color{darkblue},
  keywordstyle=\color{cyan},
  morekeywords={xmlns,version,type}% list your attributes here
}

\lstset{
  language=XML,
   literate={ć}{{\'c}}1
}
\renewcommand*{\lstlistingname}{Kod źródłowy}
% definicje kolorow
\definecolor{ciemnoSzary}{rgb}{0.15,0.15,0.15}
\definecolor{szary}{rgb}{0.5,0.5,0.5}
\definecolor{jasnoSzary}{rgb}{0.2,0.2,0.2}

% Konfiguracja verbatima
\fvset{
	frame=single,
	numbers=left,
	fontsize=\footnotesize,
	numbersep=12pt,
%	framerule=.5mm,
	rulecolor=\color{ciemnoSzary},
%	fillcolor=\color{jasnoSzary},
	framesep=4pt,
	stepnumber=1,
	numberblanklines=false,
	tabsize=2,
%	formatcom=\color{szary}
}
\newcommand{\listequationsname}{Spis wzorów}
\newcommand{\equationcaption}[1]{\begin{flushright}\emph{#1}\end{flushright}}
\newcommand{\rightcaption}[1]{\begin{flushright}\emph{#1}\end{flushright}}
\newlistof{myequations}{equ}{\listequationsname}
\newcommand{\myequations}[1]{%
\addcontentsline{equ}{myequations}{\protect\numberline{\theequation}#1}\par}

\newcommand{\listofmyalgorithmsname}{Spis algorytmów}
\newlistof{myalgorithm}{algo}{\listofmyalgorithmsname}
\newcommand{\myalgorithm}[1]{%
\addcontentsline{algo}{myalgorithm}{\protect\numberline{\thealgorithm}#1}\par}


\newcommand{\listofmyfiguresname}{Spis rysunków}
\newlistof{myfigure}{figu}{\listofmyfiguresname}
\newcommand{\myfigure}[1]{%
\addcontentsline{figu}{myfigure}{\protect\numberline{\thefigure}#1}\par}

\floatname{algorithm}{Algorytm}

\newtheorem{mydef}{Definicja}



\begin{document}


\newcommand{\resultChart}[7][140]{
\def\dataS{{#2}}
	\begin{figure}[H]
	
\centering

\begin{center}
\begin{tikzpicture}
 
\begin{axis}[
ybar,
bar width=20,
legend style={at={(0.5,-0.25)},
anchor=north,legend columns=-1},
ylabel={Wartość miary},
symbolic x coords={\dataS},
xtick=data,
height=  {#1},
width=0.8\textwidth,
ymin=0, ytick={0,0.5,1},
ymax=1.5,
nodes near coords,
nodes near coords align={vertical},
]
\addplot coordinates { (\dataS,{#3}) };
\addplot coordinates {(\dataS,{#4}) };
\addplot coordinates { (\dataS,{#5}) };
\legend{Recall,Precission,F1-Score}
\end{axis}
\end{tikzpicture}
\end{center}
\caption{{#6}}
\myfigure{{#6}}
\label{{#7}}
\end{figure}
}


\pgfkeys{/pgf/number format/use comma}
\pgfkeys{/pgf/number format/.cd, set thousands separator={}}%
\nocite{*}
\title{ Wielokryterialny problem rozmieszczenia zraszaczy wodnych na zadanej powierzchni }
\titleEN{ Multicriteria water sprinklers deployment problem on a given area}
\shortTitle{SHORT TITLE}
\author{inż. Grzegorz Dziedzic}
\advisor{dr Mariusz Fraś}
\instituteLogo{logos/pwr}
\slowaKluczowe{optymalizacja wielokryterialna,\\ algorytmy genetyczne,\\zraszacze wodne}

\date{\number\the\year}

% Wstawienie abstractu pracy
	%\input {abstract}

\abstractSH{SHORT ABSTRACT}

\abstractPL{
	ABSTRACT PL
}
\abstractEN{
	ABSTRACT EN
}

\maketitle

\textpages


\graphicspath{ {img/} }
\DeclareGraphicsExtensions{.pdf,.png,.jpg}
\chapter{Wstęp}
\section{Wprowadzenie}
Odpowiednie nawodnienie ogrodu jest jedną z podstawowych czynności pielęgnacyjnych. Gdy właściciel dysponuje odpowiednim budżetem najlepszym rozwiązaniem będzie dla niego inwestycja w automatyczny system nawadniania. System taki składa się z zraszaczy wodnych, rur pomiędzy nimi oraz systemu sterowania. Takie rozwiązanie pozwala zaoszczędzić czas tracony na ręcznym podlewaniu ogrodu oraz zapewnia równomierne nawodnienie na całej ustalonej powierzchni. Jednym z głównych problemów koniecznych do rozwiązania podczas instalacji takiego systemu jest odpowiednie rozmieszczenie poszczególnych zraszaczy. Te najczęściej znajdują się pod ziemią oraz posiadają wynurzalną głowicę. Z tego powodu raz zainstalowany zraszacz najczęściej zostaje na swoim miejscu, aż do momentu wymiany całej instalacji wodnej. Biorąc to pod uwagę rozmieszczenie zraszaczy powinno być dobrze przemyślane już podczas etapu projektowania systemu nawadniania. Projektując taki system należy przyjąć jako cel nawodnienie całości wskazanego obszaru jak najmniejszym kosztem przy przestrzeganiu wskazanych przez właściciela ograniczeń.\\
Proces projektowania sieci zraszaczy może być żmudny oraz długotrwały, biorąc pod uwagę różnorodność sprzętu dostępnego na rynku czy chociażby nieregularność powierzchni, która ma zostać nawodniona. Z pomocą może przyjść tutaj nowoczesna technologia. Opisany powyżej problem idealnie nadaje się do rozwiązania przy pomocy dostępnych algorytmów optymalizacyjnych. Praca ta będzie skupiać się na rozwiązaniu omówionego problemu poprzez opracowanie systemu wspomagania decyzji i implementację oraz porównanie wielokryterialnych algorytmów genetycznych.
\section{Cel pracy}
\section{Opis pracy}
\section{Przegląd literatury}

\chapter{Problem nawodnienia obszaru}

\chapter{Optymalizacja}
\section{Optymalizacja jednokryterialna}
\section{Optymalizacja wielokryterialna}

\chapter{Algorytmy genetyczne}
\section{Opis ogólny}
\section{Algorytmy wielokryterialne}
\subsection{NSGA-II}
\subsection{SPEA}

\chapter{Systemy wspomagania decyzji}

\chapter{Rozwiązanie problemu}
\section{System wspomagania decyzji}
\subsection{Architektura}
\subsection{Interakcja z użytkownikiem}
\section{Optymalizacja}
\subsection{Model matematyczny}
\subsection{Rezultaty}
\subsection{Porównanie algorytmów genetycznych}
\subsubsection{Plan badań}
\subsubsection{Rezultaty badań}
\subsubsection{Podsumowanie badań}

\chapter{Podsumowanie}


\begin{mydef}
\textbf{Definicja} - pierwsza
\end{mydef}




 \clearpage
\appendix
\chapter{Appendix 1}


\clearpage
\pagestyle{plain}
\listofmyfigure
\listofmyequations
\listofmyalgorithm
\clearpage

%\bibliographystyle{apalike}%Used BibTeX style is unsrt

\bibliographystyle{iisthesis}
\bibliography{bibliography}

\end{document}

